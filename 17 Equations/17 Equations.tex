\documentclass[12pt]{article} 
\usepackage{ucs} 
\usepackage[utf8x]{inputenc} 
\usepackage[english,russian]{babel}
\usepackage[left=0.2cm,right=0.4cm,top=2cm,bottom=2cm,nohead,nofoot]{geometry}
\begin{document}
\title{17 Формул, которые изменили мир}
\date{}
\maketitle
\begin{tabular}{ l l l }
	1) Теорема Пифагора & $ a^{2}+b^{2}=c^{2} $ & Пифагор 530 г. до н.э. \\
	2) Логарифм & $ \log{xy}=\log{x} + \log{y} $ & Джон Непер 1610 г. \\
	3) Формула Ньютона-Лейбница & $ \frac{d f}{d t}=\lim\limits_{h\to 0} \frac{f(t+h)-f(t)}{h}$ & Ньютон 1668 г. \\
	4) Закон Всемирного Тяготения & $ F=G\frac{m_{1}m_{2}}{r^{2}}$ & Исаак Ньютон 1667 г. \\
	5) Комплексное число & $ \iota^{2}=-1$ & Леонард Эйлер 1750 г. \\
	6) Эйлерова характеристика & $ V-E+F=2 $ & Леонард Эйлер 1751 г. \\
	7) Нормальное распередление & $ \Phi(x)=\frac{1}{\sqrt{2\pi\rho}}e^{\frac{(x-\mu)^2}{2\rho^{2}}} $ & Карл Гаусс 1810 г. \\
	8) Волновое уравнение & $ \frac{\partial^{2}u}{\partial{t}^{2}}=c^{2}\frac{\partial^{2}u}{\partial{x}^{2}} $ & Жан Лерон Д’Аламбер 1746 г. \\
	9) Преобразование Фурье & $ f(\omega)=\frac{1}{\sqrt{2\pi}} \int\limits_{-\infty}^\infty f(x) e^{-ix\omega} d{x} $ & Жан Батист Жозеф Фурье 1822 г. \\
	10) Уравнение Навера-Стокса &  $ \rho(\frac{\partial{\mathrm{v}}}{\partial{t}}+{\mathrm{v}}\cdot\nabla {\mathrm{v}})=-\nabla p+\nabla\cdot T+f $ & Анри Навье, Джордж Стокс 1845 г. \\
	11) Уравнения Максвелла & 
	 \begin{tabular}[t]{ll} 
	$ \nabla \cdot  {\bf E}=\frac{\rho}{\epsilon_{0}}c$ & $\nabla \cdot {\bf H}=\frac{1}{c} \frac{\partial E}{\partial t} $ \\ 
	$ \nabla \cdot {\bf H}=\frac{1}{c} \frac{\partial E}{\partial t}$ & $\nabla \cdot  {\bf E}=\frac{\rho}{\epsilon_{0}}c $ \\
	  \end{tabular} & Джеймс Клерк Максвелл 1865 г. \\
	12) Второй закон термодинамики & $ d S \geq 0$ & Людвиг Больцман 1874 г. \\
	13) Теория относительности & $ E=mc^{2} $ & Альберт Эйнштейн 1905 г. \\
	14) Уравнение Шрёдингера & $ \iota \hbar \frac{\partial}{\partial t} \Psi = H \Psi $ & Эрвин Шрёдингер 1927 г. \\
	15) Теория Информации & $ H = -\sum p(x) \log{p(x)} $ & Клод Шеннон 1949 г. \\
	16) Теория хаоса &  $ x_{t+1}=k x_{t} (1 - x_{t})$ &  Роберт Мэй 1975 г. \\
	17) Уравнение Блэка-Шоулза & $ \frac{1}{2} \sigma^{2} S^{2} \frac{\partial^{2} \mathrm{V}}{\partial S^{2}}+r S \frac{\partial \mathrm{V}}{\partial S} + \frac{\partial \mathrm{V}}{\partial t} - r V = 0 $ & Фишер Блэк, Майрон Шоулз 1990 г. \\

\end{tabular}
\end{document}